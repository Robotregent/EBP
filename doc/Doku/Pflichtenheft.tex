\documentclass[a4paper,10pt]{article}

\usepackage{ucs}
\usepackage[utf8x]{inputenc}
\usepackage[german]{babel}
\usepackage{fontenc}
\usepackage{graphicx}
\usepackage{fullpage}
\usepackage{hyperref}

\author{Sebastian Kumminger\\
	Fabian Schneider\\
	Tobias Himmer}

\title{Pflichtenheft\\
	Elektronisch Betreuungsplanung und Tagesdokumentation}

\date{18.11.2011}

\begin{document}

\maketitle

\tableofcontents
\newpage
\section{Produktbeschreibung}
Ziel ist es, eine Applikation zu entwickeln, die es Einrichtungen der Behinderten- und Altenhilfe ermöglicht, die langfristige Betreuungsplanung mit der Tagesdokumentation zu verknüpfen. Es sollen Ereignisse aus dem Tagesgeschehen, die in der Tagesdokumentation der Wohngruppe / des Pflegeheims erfasst werden, der Betreuungsplanung des bertoffenen Bewohners zugeordnet werden können. Damit soll doppelter Dokumentationsaufwand verhindert werden.
\subsection{Anwendungsbereiche}
\subsection{Zielgruppe}
\subsection{Betriebsbedingungen}

\section{Funktionsumfang}
\subsection{Kernfunktionen}
\subsubsection{Rechteverwaltung}
Mitarbeiter und Bewohner werden einer Wohngruppe zugeordnet. Jeder Mitarbeiter darf nur auf die Bewohner seiner Wohngruppe zugreifen.
\subsubsection{Bewohnerverwaltung}
\begin{itemize}
	\item Personenbezogene Informationen
	\item Planung von Projekten
	\begin{itemize}
		\item Verknüpfung zur Betreuungsplanung
	\end{itemize}
	\item Protokollfunktion
	\begin{itemize}
		\item Verknüpfung zur Betreuungsplanung
	\end{itemize}
	\item  Betreuungsplanung
	\begin{itemize}
		\item Betreuungsplanung nach einem aktuellen Standard der Alten- oder Behindertenhilfe
		\item Übernahme von Ereignissen aus der Tagesdokumentation in die Kategorien der Betreuungsplanung
	\end{itemize}
\end{itemize}
\subsubsection{Tagesdokumentation}
\begin{itemize}
	\item Gruppenbuch
	\begin{itemize}
		\item Gruppenbezogene Ereignisse werden mit Uhrzeit und Mitarbeiternamen erfasst
		\item Klientenbezogene Ereignisse können in die Betreuungsplanung des entsprechenden Klienten übertragen werden.
		\item Such- und Listenfunktion für die klientenbezogenen Ereignisse
		\item Meldeliste für alle Bewohner
		\begin{itemize}
			\item Urlaub, Krankheit, Abwesenheit mit Grund
			\item Export der Meldeliste in einem gängigen Format (z.B. CSV, XML)
		\end{itemize}
	\end{itemize}
\end{itemize}
\subsubsection{Multilinguale Oberfläche}
Unterstützte Sprachen
\begin{itemize}
 \item Deutsch
 \item Englisch
% \item Esperanto
% \item Klingonisch
\end{itemize}

\subsection{Zusatzfunktionen}
\subsubsection{Bewohnerverwaltung}
\begin{itemize}
	\item Adressverwaltung
	\item Planung von Aufgaben
	\item Generierung bestimmter Dokumente aus den Daten der Datenbank
	\begin{itemize}
		\item Z.B. Heimvertrag
	\end{itemize}
\end{itemize}
\subsubsection{Tagesdokumentation}
\begin{itemize}
	\item Tagesplan / Wochenplan
	\begin{itemize}
		\item Alle Aktivitäten / Aufgaben an einem bestimmten Tag / über einen bestimmten Zeitraum.
		\item Termine können als ics Termin an einen E-Mailempfänger verschickt werden
	\end{itemize}
\end{itemize}

\section{Produktumgebung}
\subsection{Softwareanforderungen}
\begin{itemize}
 \item Betriebssystem: aktuelle GNU/Linux Distribution;
 \item Datenbank: MySql Server;
 \begin{itemize}
  \item Als Datenbank wird MySQL eingesetzt. Diese ist für nicht-kommerzielle Zwecke (im Sinne der GPL) frei erhältlich und sehr weit verbreitet. 
Ein Wechsel auf eine andere Datenbank (zum Beispiel PostgreSQL) sollte dank ODB ohne größere Probleme möglich sein.
 \end{itemize}
 \item Bibliotheken: Qt;
\end{itemize}
\subsection{Hardwareanforderungen}
Server:
Workstation:
\subsection{Anforderungen an die Organisation (Orgware)}

\section{Anforderungen an die Entwicklungsumgebung}
\subsection{Softwareanforderungen}
\begin{itemize}
 \item Datenbank: MySql-Client Entwicklungsbibliothek;
 \item GUI: Qt-SDK; 
 \item ORM: ODB-SDK;
 \item Buildtools: CMake,  gcc;
\end{itemize}
\subsection{Object-Relational-Mapper}
Um die Entwicklungsdauer zu verkürzen und die Wartbarkeit der Persistenzschicht zu erhöhen, wird ein Object Relational Mapper (ORM) eingesetzt. 
Dieser hilft die relationale Tabellenstruktur der Datenbank auf Objekte der Programmiersprache abzubilden.
Das verwendete Qt4-Toolkit enthält war eine plattformunabhängige Schnittstelle für Datenbanken (QtSql), jedoch hat diese keine ORM funktion.
Folgende ORM-Systeme wurden in Betracht gezogen \newline\url{http://en.wikipedia.org/wiki/List_of_object-relational_mapping_software}:
\begin{itemize}
	\item LiteSQL - Datenbankschema wird mit XML Dateien beschrieben
	\item ODB - Datenbankschema wird aus den C++ Klassen generiert
	\item QxOrm - Benutzt QtSql; Unter Umständen müssen direkte Aufrufe von QtSql erfolgen ("QxOrm cannot resolve all problems with sql and databases, so it is sometimes necessary to use QtSql engine of Qt library to write your own sql query or stored procedure." \url{http://www.qxorm.com/qxorm_en/tutorial.html})
\end{itemize}
Alle Systeme bieten eine Qt Integration, die die Schnittstelle zwischen GUI und Persistenzschicht vereinfachen. Ebenso unterstützen alle Systeme mehrere Datenbankbackends.
Letztendlich fiel die Entscheidung auf ODB, da diese Bibliothek (unter anderem durch Caching) die beste Performance verspricht. Ausserdem lassen sich Klassen mit geringem Aufwandt persistent machen.


\section{Grafische Benutzeroberfläche}
\subsection{Grundlegende Anforderungen}
Um die Usability und das Look-and-Feel einer Desktopanwendung zu gewährleisten, wird ein GUI Toolkit verwendet.
Hierfür soll Qt4 zum Einsatz kommen. Qt ist zum einen portabel auf verschiedene Plattformen 
und bietet neben den grafischen Bedienelementen eine breite Auswahl an Schnittstellen zu Systemfunktionen.
\subsection{Anforderungen an die Verknüpfung der Betreuungsplanung}
Eine der Hauptfunktionen soll das Zuordnen von Tagesereignissen zu den Zielen der Betreuungsplanung sein. Diese Funktion soll über 
kurze und intuitive Wege erreichbar sein. Die Usability dieser Funktion steht im Vordergrund.
\subsection{Mockup}

\section{Qualitätssicherung}
\subsection{Usability}
Die Usability der grafischen Bedienoberfläche soll in zwei Stufen getestet werden.
\begin{itemize}
\item Es wird ein horizontaler Prototyp erstellt,  um die grafischen Bedienelemente auf verschiedene Anwendungsfälle vorab zu testen. 
\item Für die funktionsfähige Anwendung wird der Alltagsgebrauch simuliert werden, um nicht erkannte Anwendungsfälle oder Fehlverhalten zu entdecken.
\end{itemize}
Die Probanden der Usability-Tests sollen aus der Zielgruppe sein, um besondere Bedürfnisse erkennen zu können.    
\subsection{Codequalität}
Die Codequalität soll durch Unit Tests gewährleistet werden.
Die Unittests werden mit Hilfe der QtTest Bibliothek und dem CUnit Framework entwickelt.
Zusätzlich wird, um Programmierfehler frühzeitig aufzuspüren, ein Speicherdebugger (valgrind) eingesetzt.

\section{Nichtfunktionale Anforderungen}
%\subsection{Barrierefreiheit}
\subsection{Datensicherheit}
\subsection{Lizenzen}
Die Software wird unter GPLv3 lizensiert und veröffentlicht.
\section{Termine}

\end{document}
