\documentclass[a4paper,10pt]{article}
\usepackage[utf8x]{inputenc}
\usepackage{ucs}
\usepackage[german]{babel}
\usepackage{fontenc}
\usepackage{graphicx}
\usepackage{tabularx}		% gescheite Tabellen
\usepackage{fullpage}		% weniger rand
\usepackage{hyperref}
\usepackage{url}
\usepackage[numbers]{natbib}	% literaturliste
\usepackage{listings}		% codelistings
\usepackage{color}
\usepackage{xspace}
\usepackage{longtable, lscape}

\newcommand\EBP{\textit{Elektronische Betreuungsplanung}\xspace}
%opening
\title{Gliederung \\ Elektronische Betreuungsplanung}
\author{Sebastian Kumminger \\
	Tobias Himmer \\
	Fabian Schneider}


\begin{document}

\maketitle

\section{Projektablauf}
\subsection{Anforderungsdefinitionen}
\subsubsection{User Stories}
\subsection{Scrum}

\section{Technische Realisierung}
\subsection{Eingesetzte Entwurfsmuster}
\subsection{Architektur}
\subsubsection{Aufteilung der Module}
\subsubsection{Buildsystem}
\subsection{Qt als Grafikbibliothek}
\subsection{Persistenzschicht mit ODB (libEBPdb)}
\subsubsection{Object-Relational-Mapping (ORM)}
\subsubsection{Datenbankunabhängigkeit}
\subsubsection{Code-Generierung}
\subsubsection{Integration in das Buildsystem}
\subsubsection{Datenbankschema}
\subsubsection{Abstraktion der Schnittstelle}

\section{Qualitätssicherung}
\subsection{Usability}
\subsubsection{Thinking-Aloud-Tests}
\subsection{Codequalität} 
\subsubsection{Aufbau Unittests}

\section{Abschließende Betrachtung}
\subsection{Probleme der Persistenzschicht}
\subsection{Verbesserte Funktionalität}

\end{document}
