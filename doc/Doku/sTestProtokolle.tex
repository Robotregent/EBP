\section{Ergebnissprotokolle Usability Tests}
\subsection{Tests zum Kapieren}
\paragraph*{Test Eins}
Proband: Studentin - Lehramt. 
\begin{itemize}
	\item Kalender PopUp bei Auswählfelder für ein Datum.
	\item Die verschiedenen Protokolle durch ihr Datum auswählen.
	\item Im Textfeld für den Teilnehmer erscheint nicht automatisch ein Cursor.
	\item Zweck der Checkbox Schriftführer war nicht gleich ersichtlich.

\end{itemize}

\paragraph*{Test Zwei}
Proband: Student - Informatik. 
\begin{itemize}
	\item Empfindet das Docking an allen Positionen als störend. Sollte eingeschränkt werden.
	\item Empfindet Ereignisliste als unübersichtlich.
	\item Vermisst Logoutfunktion.
	\item Vermisst Speicherknopf.
\end{itemize}

\subsection{Tests mit Schlüsselaufgaben}
\paragraph*{Test Drei}
Proband: Sozialarbeiterin - arbeitet auf einer Wohngruppe mit Menschen mit Behinderungen. 
\begin{itemize}
	\item Betreuung in gesetzliche Betreuung umbenennen. 
	\item Betreuungsdokumentation überall gleich nennen.
	\item Ereignisse sollten Gruppenbuch heißen.
\end{itemize}

\paragraph*{Test Vier}
Proband: Studentin - Lehramt.
\begin{itemize}
	\item Baumstuktur der Navigation nicht sofort erkennbar.
	\item Funktion des Texttransfers war nicht sofort klar.
	\item Das Speicherkonzept war nicht sofort klar.
\end{itemize}

\paragraph*{Test Fünf}
Proband: Student - Soziale Arbeit. 
\begin{itemize}
	\item Wünscht sich Nachfrage, ob bei Maskenwechsel gespeichert werden soll. 
	\item Zeilenumbruch bei Texttransfer fehlt. 
	\item Hilfe Button umplazieren.
	\item Bessere Beschriftung: "Markierter Text kopieren"
\end{itemize}

\paragraph*{Test Sechs}
Proband: Studentin - Lehramt. 
\begin{itemize}
	\item  
	\item 
\end{itemize}
