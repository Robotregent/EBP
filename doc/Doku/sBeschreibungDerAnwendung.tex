\section{Beschreibung der Anwendung}
\label{sec:beschreibung}
\subsection{Admindialog}
\paragraph{Wohngruppe}\mbox{}\\
Im \textit{Admindialog} können Administratoren Bewohner, Wohngruppen/-heime und Mitarbeiter verwalten.
\begin{figure*}[h]
	\begin{center}
		\includegraphics[keepaspectratio=true, width=0.85\textwidth]{pics/admin3.png}
		\caption{Wohngruppe}
		\label{Admindialog Wohngruppe}
		\caption{Graphen eines Interfaces}
		\label{Admindialog_Mitarbeiter_erstellen}
	\end{center}
\end{figure*}
\FloatBarrier
\noindent
Hier werden Wohngruppen erstellt, diese dienen als Gruppierung für sowohl Bewohner, als auch für Mitarbeiter.
\newpage
\noindent
\paragraph{Bewohner}\mbox{}\\
Bewohner werden mit einer Bewohnernummer, Vor-/Nachnamen und ihrer Wohngruppe  erstellt. 
\begin{figure*}[h]
	\begin{center}
		\includegraphics[keepaspectratio=true, width=0.85\textwidth]{pics/admin2.png}
		\caption{Bewohner}
		\label{Admindialog Bewohner}
		\caption{Graphen eines Interfaces}
		\label{Admindialog_Bewohner}
	\end{center}
\end{figure*}
\FloatBarrier
\noindent
Restliche Informationen werden im Client von zugewiesenen Bezugsbetreuern ausgefüllt.
\newpage
\noindent
\paragraph{Mitarbeiter}\mbox{}\\
Mitarbeiter werden mit Login sowie einigen persönlichen Daten, Name und Kontaktmöglichkeiten, sowie ihrer Berechtigung erstellt. Es ist darauf zu achten das jedem Mitarbeiter beim erstellen mindestens eine Wohngruppe zugeordnet werden muss, die er betreut. Er kann allerdings natürlich auch für mehrere Wohngruppen zuständig sein und auch der Bezugsbetreuer für Bewohner sein.
\begin{figure*}[h]
	\begin{center}
		\includegraphics[keepaspectratio=true, width=0.85\textwidth]{pics/admin1.png}
		\caption{Mitarbeiter}
		\label{Admindialog Mitarbeiter}
	\end{center}
\end{figure*}
\FloatBarrier
\noindent
Mitarbeiter die keine Administratorenrechte haben können nur Informationen von Bewohner in ihrer Wohngruppe einsehen und nur von Bewohner deren Bezugsbetreuer sie sind verändern.
\newpage
\subsection{\EBP Client}
Der \EBP \textit{Client} besitzt eine widgetbasierende GUI die aus folgenden Elementen besteht.
\begin{itemize}
	\item einer Menüleiste\mbox{}\\
	\noindent
	Hier wird der Inhalt des Hauptfensters gespeichert, die zusätzlichen Widgets aus-, bzw eingeblendet oder der Mitarbeiter ausgeloggt.\\ Der Speichern Button ist ausgegraut wenn der eingeloggte Mitarbeiter kein Bezugsbetreuer des aktive Bewohners ist.
	\begin{figure*}[h]
		\begin{center}
			\includegraphics[keepaspectratio=true, width=0.55\textwidth]{pics/client_header.png}
			\caption{Menueleiste}
			%\label{Menüleiste, fixiert an der oberen Seite des Programms}
		\end{center}
	\end{figure*}
	\FloatBarrier
	\noindent
	\item einem Informationswidget\mbox{}\\
	Hier der momentan ausgewählte Bewohner und dessen Wohngruppe angezeigt, zum ändern öffnet sich, bei Auswahl des jeweiligen Buttons, ein Popup mit allen Wohngruppen für die der Mitarbeiter berechtigt ist, bzw. alle Bewohner der jeweiligen Wohngruppe.
	\begin{figure*}[h]
		\begin{center}
			\includegraphics[keepaspectratio=true, width=0.85\textwidth]{pics/client_info.png}
			\caption{Informationswidget}
			%\label{Bewohner- und Stationswidget}
		\end{center}
	\end{figure*}
	\FloatBarrier
	\newpage
	\item einem Navigationswidget\mbox{}\\
	\noindent
	Im Navigationsmenü wird der Inhalt des Hauptfensters ausgewählt. Das ``Bewohner'' Tab enthält alle Daten und Menü's die sich auf den aktiven Bewohner beziehen, während das ``Wohngruppen'' Tab Unterpunkte beinhaltet die für die komplette aktuelle Wohngruppe gültig sind.	
	\begin{figure*}[h]
		\begin{center}
			\includegraphics[keepaspectratio=true, width=0.35\textwidth]{pics/client_navi.png}
			\caption{Navigationswidget}
			%\label{Navigationsmenü}
		\end{center}
	\end{figure*}
	\FloatBarrier
	\item dem Hauptfenster\mbox{}\\
	\noindent
	Das eigentliche Hauptfenster dient zur Ein-/Ausgabe von Daten und bezieht sich immer auf den im Informationswidget ausgewählten Bewohner, bzw die ausgewählte Wohngruppe. Um hier eingegebene Daten zu sichern muss der Speichern-Knopf in der Menüleiste betätigt werden.\\
	\begin{figure*}[h]
		\begin{center}
			\includegraphics[keepaspectratio=true, width=0.85\textwidth]{pics/client_main.png}
			\caption{Hauptfenster}
			%\label{Hauptfenster}
		\end{center}
	\end{figure*}
	\FloatBarrier
\end{itemize}
\subsubsection{Bewohnerbezogene Menüs}
\paragraph{Person}\mbox{}\\
Hier können Mitarbeiter die persönlichen Daten von Bewohnern einsehen.
\subsection{Lokalisierung}
Sowohl der AdminDialog als auch die \EBP sind koplett lokalisierbar. Qt stellt dafür einen einfachen Mechanismus zur Verfügung. Jeder zu
lokalisierende String wird dabei von einem Makro umschlossen. Mit Hilfe des Qt Linguist können diese Strings zu einem Wörterbuch zusammengefasst und
übersetzt werden. Für die \EBP war im Plichtenheft angedacht, ein englisches und eine deutsches Wörterbuch zur Verfügung zu stellen. In den
Gesprächen zur Anforderungsdefinition mit Herrn Zimmer zeichnete sich ein anderer Lokalisierungsbedarf an die \EBP ab.\\
Die Fachkräfte in der Altenhilfe und der Behindertenhilfe haben ihre eigene Fachsprache im täglichen Umgang mit ihren Klienten. So liegt der Fokus in
der Altenhilfe eher auf pflegerischen Tätigkeiten, wohingegen in der Behindertenhilfe pädagogische Ziele und Maßnahmen im Fordergrund stehen. Ein
Klient wird in der Altenhilfe eher als Patient bezeichnet, in der Behindertenhilfe spricht man häufiger von einem Bewohner. Um diesen Umstand zu
berücksichtigen, wurden keine Wörterbücher für Englisch und Deutsch erstellt, sondern Wörterbücher für die Altenhilfe und für die Behindertenhilfe.\\
Wird der Client oder der AdminDialog ohne Parameter gestartet, wird das Wörterbuch der Behindertenhilfe geladen. Um die Lokalisierung der Altenhilfe
zu aktivieren, muss der entsprechenden Anwendung der Parameter ``-a'' übergeben werden.

\newpage

\subsection{Erstellen der Anwendung}
\subsubsection{Abhängigkeiten}
\paragraph{Zur Laufzeit:}
\begin{itemize}
	\item \textbf{odb} - ODB Laufzeitbibliothek
	\item \textbf{odb-mysql} - MySQL Backend für ODB
	\item \textbf{odb-qt} - QT integration für ODB
	\item \textbf{qt} - QT Framework
\end{itemize}
\paragraph{Zur Compilezeit:}
\begin{itemize}
	\item \textbf{CMake} - Buildsystem
	\item \textbf{ODB Toolchain} - Enthält den Precompiler
	\item \textbf{GCC} - GNU Compiler Collection
\end{itemize}
\subsubsection{Kompilieren der Quellen}
Die komplette Anwendung kann im Wurzelverzeichnis des Projekts gebaut werden:\\
\begin{lstlisting}
$ cmake .
\end{lstlisting}
Generiert das benötigte Makefile.\\
Ist dies erfolgreich abgeschlossen, wird mit\\
\begin{lstlisting}
$ make
\end{lstlisting}
der eignetliche Kompiliervorgang gestartet.\\
\subsubsection{Vorbereiten der Datenbank}
Um die Datenbank zu initialisieren befindet sich ein Shell-Script im EBPdb Verzeichnis:\\
\begin{lstlisting}
$ ./initDB.sh -u root -p "DATENBAKNAME"
\end{lstlisting}
