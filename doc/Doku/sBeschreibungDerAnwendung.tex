\section{Beschreibung der Anwendung}
\label{sec:beschreibung}

\subsection{Erstellen der Anwendung}
\subsubsection{Abhängigkeiten}
\paragraph{Zur Laufzeit:}
\begin{itemize}
	\item \textbf{odb} - ODB Laufzeitbibliothek
	\item \textbf{odb-mysql} - MySQL Backend für ODB
	\item \textbf{odb-qt} - QT integration für ODB
	\item \textbf{qt} - QT Framework
\end{itemize}
\paragraph{Zur Compilezeit:}
\begin{itemize}
	\item \textbf{CMake} - Buildsystem
	\item \textbf{ODB Toolchain} - Enthält den Precompiler
	\item \textbf{GCC} - GNU Compiler Collection
\end{itemize}
\subsubsection{Kompilieren der Quellen}
Die komplette Anwendung kann im Wurzelverzeichnis des Projekts gebaut werden:\\
\begin{lstlisting}
$ cmake .
\end{lstlisting}
Generiert das benötigte Makefile.\\
Ist dies erfolgreich abgeschlossen, wird mit\\
\begin{lstlisting}
$ make
\end{lstlisting}
der eignetliche Kompiliervorgang gestartet.\\
\subsubsection{Vorbereiten der Datenbank}
Um die Datenbank zu initialisieren befindet sich ein Shell-Script im EBPdb Verzeichnis:\\
\begin{lstlisting}
$ ./initDB.sh -u root -p "DATENBAKNAME"
\end{lstlisting}
