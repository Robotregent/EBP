\section{Technische Realisierung}

\subsection{Eingesetzte Entwurfsmuster}

\subsection{Architektur}
\subsubsection{Aufteilung der Module}
\subsubsection{Buildsystem}

\subsection{Qt als Grafikbibliothek}

\subsection{Persistenzschicht mit ODB (libEBPdb)}
\subsubsection{Object-Relational-Mapping (ORM)}
Um die Daten einer relationellen Datenbank auf Objekte innerhalb einer Programmiersprache abzubilden gibt es prinzipiell zwei Ansätze.
Entweder wird der Code der die Eigenschaften eines Objekts aus der Datenbank liest und schreibt manuell implementiert, oder aber es wird ein Mechanismus eingesetzt, der automatisiert Variablen eines Objekts den Feldern einer Datenbank zuordnet.
\subsubsection{Datenbankunabhängigkeit}
\subsubsection{Code-Generierung}
\subsubsection{Integration in das Buildsystem}
\subsubsection{Datenbankschema}
\subsubsection{Abstraktion der Schnittstelle}
