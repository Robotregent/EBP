\section{Abschließende Betrachtung}
\label{sec:abschluss}

\subsection{Probleme der Persistenzschicht}
\begin{itemize}
	\item Während der Entwicklung ist aufgefallen, dass der starke Templategebrauch von \textit{ODB} den Buildprozess deutlich spürbar verlangsamt - Obwohl das Buildsystem lediglich die Dateien neu kompiliert, die tatsächlich von Änderungen betroffen sind.\\
	\item Die Templatenutzung schränkt oft die Designmöglichkeiten ein:\\
		Da bei der Instanzierung einer Templateklasse die komplette Implementierung bekannt sein muss,
		kann es zu Problemen kommen wenn zwei Template Klassen die Methoden der jeweiligen anderen Klasse verwenden wollen.
\end{itemize}
\subsection{Fehlender Controller auf den Masken}
Der reine Model/View Ansatz von Qt ist in sehr vielen Fällen sicher ausreichend, aber vor allem das Speicherkonzept hätte von einem dezidierten
Controller profitiert. Wie die Usability-Tests gezeigt haben, erwarten die User mehr Rückmeldung als wir anfangs erwartet hatten. Als Reaktion auf
die Ergebnisse der Usability-Tests wurde nachträglich eine Rückfrage an den User eingebaut. Damit wird prüft, ob ausstehende Änderungen vor
einem Wechsel der Maske gespeichert werden sollen. Die Logik für diese Rückfrage sollte korrekter Weise in einen Controller gekapselt werden. 

\subsection{Weitere Epics}
Aus den Gesprächen mit Herrn Zimmer ergaben sich noch weitere Funktionen, in der Sprache von \textit{Scrum} Epic genannt, die in folgenden
Iterationen Implementiert werden sollten:
\begin{description}
 \item[Kalendermodul] Dieses Modul soll auf Wohngruppenebene eingeordnet werden und zur Planung des Ablaufs einer Wohngruppe/Station dienen.
Kalenderfunktionen, wie Serientermine und Ressourcenplanung sollen Teil des Moduls sein.
 \item[Medikationsplan] Jedem Bewohner soll ein Medikationsplan zugeordnet werden. Dort werdensowohl die Menge und Art der Medikation, wie auch der
verabreichende Mitarbeiter und das Datum der Medikamentengabe protokolliert.
\end{description}

 