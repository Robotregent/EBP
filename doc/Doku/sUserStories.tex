\section{User Stories}
User Stories sind ein Ansatz aus der Agilen Softwareentwicklung, um die Anforderungen verschiedener User an die Software zu definieren. Die
Formulierung einer User Story erfolgt nach folgendem Template \cite{Wikipedia_User_Story}:
\begin{lstlisting}
As a <role>, I want <goal/desire> so that <benefit>
\end{lstlisting}
Oder in der verkürzten Form:
\begin{lstlisting}
As a <role>, I want <goal/desire>
\end{lstlisting}
User Stories, die thematisch im Zusammenhang stehen werden unter einem Epic zusammengefasst. Die Sprache sollte bei der Formulierung
 von Epic und User Stories aus der Lebenswelt des Kunden stammen. Die Anforderungen an die \EBP wurden in folgende Epic und User Stories aufgeteilt:
\newline
\begin{tabular}{|p{0.3\textwidth}|p{0.7\textwidth}|}
  \hline
  \textbf{Epic} & \textbf{User Story}  \\
  \hline
  
  \hline
\end{tabular}
