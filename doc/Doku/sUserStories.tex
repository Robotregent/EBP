\section{User Stories}
User Stories sind ein Ansatz aus der Agilen Softwareentwicklung, um die Anforderungen verschiedener User an die Software zu definieren. Die
Formulierung einer User Story erfolgt nach folgendem Template \cite{Wikipedia_User_Story}:
\begin{lstlisting}
As a <role>, I want <goal/desire> so that <benefit>
\end{lstlisting}
Oder in der verkürzten Form:
\begin{lstlisting}
As a <role>, I want <goal/desire>
\end{lstlisting}
User Stories, die thematisch im Zusammenhang stehen werden unter einem Epic zusammengefasst. Die Sprache sollte bei der Formulierung
 von Epic und User Stories aus der Lebenswelt des Kunden stammen. Die Anforderungen an die \EBP wurden in folgende Epic und User Stories aufgeteilt:
\newline

\begin{longtable}{|p{0.3\textwidth} | p{0.7\textwidth}|}
  \hline
  \textbf{Epic} & \textbf{User Story}  \\
  \hline
  Die persönlichen Daten eines Klienten sollen digital verwaltet werden & \\
  \hline
 $ \longmapsto $& Als Pflegekraft kann ich die Kontaktdaten eines Klienten anzeigen lassen und auch bearbeiten \\
  \cline{2 -2}
  $ \longmapsto $ & Als Pflegekraft kann ich Informationen über die Leistungsträger der Klienten in meinem Verantwortungsbereich anzeigen lassen und
auch bearbeiten. \\
   \cline{2 -2}
 $ \longmapsto $ & Als Pflegekraft kann ich für die Klienten in meinem Verantwortungsbereich anzeigen lassen, welche freiheitseinschränkenden
Maßnahmen richterlich angeordnet werden, um Rechtssicherheit bei meiner Arbeit zu erlangen. \\
   \cline{2 - 2}
  $ \longmapsto $ & Als Pflegekraft kann ich die gesetzliche Betreuung und die institutionelle Bezugsbetreuung anzeigen lassen und auch bearbeiten. \\
  \hline
  Für jeden Klienten können mehrere Projekte organisiert werden, die eine pädagogische Zielsetzung haben & \\
  \hline
$ \longmapsto $ & Als für ein Projekte verantwortlicher Mitarbeiter kann ich neue Projekte anlegen \\
 \cline{2 - 2}
  $ \longmapsto $  & Als für ein Projekte verantwortlicher Mitarbeiter kann ich pädagogische Ziele für ein Projekt definieren \\
   \cline{2 - 2}
 $ \longmapsto $ & Als Pflegekraft kann ich Textfragmente aus der Zielsetzung oder der Projektbeschreibung in einen bestimmten
Bereich der Betreuungsdokumentation übertragen, um den Fortschritt und die Zielerreichung der Langzeitplanung einfach dokumentieren zu können. \\
 \hline
  Bei Besprechungen mit einem Klienten müssen Protokolle erstellt werden. Diese Protokolle sollen Teil der \EBP sein. & \\
 \hline
  $ \longmapsto $ & Es können für jeden Klienten mehrere Projekte angelegt werden. \\
\cline{2 - 2}
 $ \longmapsto $ & Bei jedem Protokoll sind die Teilnehmer und das Protokolldatum ersichtlich. \\
\cline{2 - 2}
 $ \longmapsto $ & Ein oder mehrere Teilnehmer können als Protokllant markiert werden \\
\cline{2 - 2}
 $ \longmapsto $ & Als Pflegekraft kann ich Textfragmente aus dem Protokolltext in einen bestimmten
Bereich der Betreuungsdokumentation übertragen, um den Fortschritt und die Zielerreichung der Langzeitplanung einfach dokumentieren zu können. \\
 \hline
 Für jeden Klienten solle es eine Betreuungsdokumentation für verschiedene Lebensbereiche geben. & \\
  \hline
  $ \longmapsto $ & Als Bezugsbetreuer kann ich in jedem Lebensbereich den Hilfebedarf eines Klienten kategorisieren. \\
\cline{2 - 2}
 $ \longmapsto $ & Als Bezugsbetreuer kann ich in jedem Lebensbereich den Hilfebedarf eines Klienten prosaisch näher beschreiben. \\
\cline{2 - 2}
 $ \longmapsto $ & Als Bezugsbetreuer kann ich in jedem Lebensbereich pädagogische Ziele definieren, um eine zielgerichtete pädagogische Arbeit zu
ermöglichen. \\
\cline{2 - 2}
 $ \longmapsto $ & Als Pflegekraft kann ich Textfragmente aus anderen Teilen der \EBP an die prosaische Beschreibung der einzelnen Lebensbereiche
senden, um den Fortschritt und die Zielerreichung der Langzeitplanung einfach dokumentieren zu können. \\
 \hline
  Es gibt ein Gruppenbuch, in dem die tagesaktuellen Geschehnisse einer Wohngruppe dokumentiert werden. & \\
  \hline
  $ \longmapsto $ & Als Pflegekraft kann ich ein Ereignis auf einer Wohngruppe in meinem Verantwortungsbereich mit de \EBP dokumentieren, um eine
lückenlose Kommunikation mit meinen Kollegen zu gewährleisten. \\
\cline{2 - 2}
 $ \longmapsto $ & Als Pflegekraft kann ich Textfragmente aus einem Ereignis in einen bestimmten
  Bereich der Betreuungsdokumentation übertragen, um den Fortschritt und die Zielerreichung der Langzeitplanung einfach dokumentieren zu können. \\
 \hline
 Die Abwesenheit eines Klienten von der Wohngruppe über einen Zeitraum größer gleich einem Tag muss zur Abrechnung mit dem Leistungsträger
dokumentiert werden. \\
  $ \longmapsto $ & Als Pflegekraft kann ich die Abwesenheit eines Klienten mit der \EBP dokumentieren, um eine der Verwaltung eine einfache
Abrechnung zu ermöglichen. \\
 \hline
\end{longtable}
